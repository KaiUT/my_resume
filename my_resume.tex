%%%%%%%%%%%%%%%%%%%%%%%%%%%%%%%%%%%%%%%%%
% Resume
% LaTeX Template
% Version 2.0 (8/5/13)
%
% This template has been downloaded from:
% http://www.LaTeXTemplates.com
%
% Original author:
% Trey Hunner (http://www.treyhunner.com/)
%
% Important note:
% This template requires the resume.cls file to be in the same directory as the
% .tex file. The resume.cls file provides the resume style used for structuring the
% document.
%
%%%%%%%%%%%%%%%%%%%%%%%%%%%%%%%%%%%%%%%%%

%----------------------------------------------------------------------------------------
%	PACKAGES AND OTHER DOCUMENT CONFIGURATIONS
%----------------------------------------------------------------------------------------

\documentclass{my_resume} % Use the custom resume.cls style

\usepackage[left=0.7cm,top=0.7cm,right=0.9cm,bottom=0.7cm]{geometry} % Document margins
\newcommand{\tab}[1]{\hspace{.2667\textwidth}\rlap{#1}}
\newcommand{\itab}[1]{\hspace{0em}\rlap{#1}}
\renewcommand{\familydefault}{\sfdefault}
\name{Kai Liu} % Your name
\address{\href{mailto:kai.liu@utexas.edu}{kai.liu@utexas.edu}
  \textbullet{} {512.917.6781} \textbullet{}
  GitHub://\href{https://github.com/KaiUT}{KaiUT} \textbullet{}
  LinkedIn://\href{https://www.linkedin.com/in/kai-liu-27200774}{Kai Liu}
}

\begin{document}

%----------------------------------------------------------------------------------------
%	PROFILE
%----------------------------------------------------------------------------------------

\begin{rSection}{Profile}
% \begin{rSubsection}{ }{}{ }
\renewcommand\labelitemi{$\cdot$}
\begin{itemize}
  \item Ph.D. in Computational Biology (expected August 2018);
  \vspace{-1.7em}\\
  \item Ample Skills in data science:
  \vspace{-1.7em}\\
    \begin{itemize}
      \item Experience in data mining, machine learning, statistical inference,
        big data analysis, and natural language processing;
      \vspace{-1.7em}\\
      \item Comprehensive technical/computing skills including Python, R, Git,
        C++, SQL, Hadoop, Spark;
      \vspace{-1.7em}\\
    \end{itemize}
  \item Excellent problem solving skills in both independent and team
    environments;
  \vspace{-1.7em}\\
  \item Skilled presenter of technical materials to both technical and
    non-technical audiences;
  \vspace{-1.7em}\\
  \item Quick, thorough and effective learner.
\end{itemize}
% \end{rSubsection}
\end{rSection}

%----------------------------------------------------------------------------------------
%	EDUCATION SECTION
%----------------------------------------------------------------------------------------

\begin{rSection}{Education}

{\bf Ph.D. in Computational Biology} {$\diamond$} {\em The University of
Texas at Austin, Austin, TX} \hfill {Expected August 2018} \\
% \vspace{0.3em}\\
{\bf M.S. in Microbiology} {$\diamond$} {\em Huazhong Agricultural University,
Wuhan, China} \hfill {Grad. June 2013} \\
% \vspace{0.3em}\\
{\bf B.S. in Biotechnology} {$\diamond$} {\em Huazhong Agricultural University,
Wuhan, China} \hfill {Grad. June 2011}
\end{rSection}

%----------------------------------------------------------------------------------------
%	RESEARCH EXPERIENCE SECTION
%----------------------------------------------------------------------------------------

\begin{rSection}{Research Experience}

\begin{rSubsection}{Graduate Research Assistant}{The University of Texas at
  Austin}{December 2014 - Present}
  \textbf{Developing Infectious Diseases Surveillance App $|$ Python}
  \item Retrieved and cleaned infectious diseases related data from Google
    Trends, Wikipedia, WordPress etc;
  \item Developed a \underline{regression model} and a \underline{Multivariate
      Exponentially Weighted Moving Average (MEWMA)} model to detect infectious
      disease outbreaks using multiple data sources, in collaboration with a
      mathematician;
  \item Optimizing data sources on infectious diseases surveillance in
    different regions (554 time-series) by combining above models and
    \underline{stepwise variable selection algorithms};
  \item Connecting algorithms with the App back-end and front-end, and
    integrating the App into Cloud Ecosystem, in collaboration with a front-end
    engineer.

  \textbf{Assessed Real-time Zika Risk in the State of Texas $|$ R}
  \item Collaborated with other researchers in developing a \underline{branching
      process model framework} that captures variation and uncertainty in Zika
      case reporting, importations, and transmission;
  \item Applied the framework to assess county-level epidemic risk throughout
    Texas.

  % \underline{Developed Mathematical Models for Infectious Diseases}
  % \item Developed Ordinary Differential Equations framework for infectious diseases
    % eliminating the assumption that all individuals in a population have the
    % same number of contacts, based on models published previously.
  % \item Simulated infectious diseases spreading on contact networks using
    % multiple algorithms.
% \item Publication in preparation.
\end{rSubsection}
\end{rSection}

%----------------------------------------------------------------------------------------
% Course Projects
%----------------------------------------------------------------------------------------

\begin{rSection}{Projects}

  \begin {rSubsection}{Developing a R Package for Big Data Analysis $|$ R \&
    Rcpp}{}{}
  \item Implementing following algorithms in the package: Stochastic gradient
    descent using line search and quasi-Newton methods to determine step size
    {$\cdot$} The lasso {$\cdot$} The proximal gradient method {$\cdot$}
    Laplacian smoothing solved by sparse Cholesky/LU, the Gauss-Seidel method,
    the Jacobi iterative method, and conjugate gradient method {$\cdot$} Graph
    fused lasso solved by Alternating Direction Method of Multipliers (ADMM)
    {$\cdot$} Sparse matrix factorization.
  \end{rSubsection}

  \begin{rSubsection}{Predicted the Direction of Exchange-Traded Fund
    (ETF) movement $|$ Python}{}{}
  \item Retrieved nine histrorical ETF sector datasets from Yahoo Finance;
  \item Implemented \underline{Logistic regression}, \underline{Ridge \& Lasso
    regression}, and \underline{Artificial Neural Network} to predict the
    direction of ETF movement;
  \item Achieved an \underline{accuracy of 55\% $\sim$ 60\%} for predicting
    nine ETF sectors movement; and the trading strategy based on my prediction
    \underline{outperforms} baseline strategies.
  \end{rSubsection}

  \begin{rSubsection}{Denoised GPS Data by Applying Kalman Filter $|$ R}{}{}
  \item Implemented \underline{Kalman filter}, and smoothed GPS data collected from
    a vehicle cruising around campus (814458 samples).
  \end{rSubsection}

  \begin{rSubsection}{Predicted Yelp Rating Based on User Review Enhanced
    Collaborative Filtering $|$ R}{}{}
  \item Extracted user opinions from restaurants dataset from Yelp ($\sim$10GB)
    using \underline{Stanford coreNLP tool};
  \item Developed a \underline{new Collaborative Filtering-based method} to
    improve the accuracy of user's rating prediction and solve the sparseness
    of dataset by combining item's features and user opinions from all reviews;
  \item \underline{Improved the prediction accuracy by 4.23\%} compared to the
    traditional KNN method, and the \underline{coverage is 100\%}.
  \end{rSubsection}

% \begin{rSubsection}{Forecasted Tourism Earnings of United Kingdom}{}{October
  % 2015 - December 2015}
  % \item Predicted Tourism Earnings of UK using a dynamic linear
    % regression model and Forward Filtering and Backward Sampling algorithm.
% \end{rSubsection}

% \begin{rSubsection}{Statistical Modeling}{}{November 2014 - December 2014}
  % \item Applied a regression model to a dataset to determine 1) factors related
  % to 12 month weight loss, and 2) whether an intervention was effective in
  % increasing weight loss by using both frequentist and Bayesian inference
  % methods.
% \end{rSubsection}
\end{rSection}

%----------------------------------------------------------------------------------------
% Skills
%----------------------------------------------------------------------------------------

\begin{rSection}{Skills}

\begin{tabular}{ @{} >{\bfseries}l @{\hspace{6ex}} l }
  Programming & Fluency in Python(NumPy, SciPy, Matplotlib, pandas,
  scikit-learn), R, Git {$\cdot$} Familiar with \\
  & MATLAB, Linux, LaTex {$\cdot$} Experience in C++, SQL, Hadoop, Spark\\
  Data Mining \& Machine Learning & Regression with regularization {$\cdot$}
  Neural Network {$\cdot$} Support Vector Machine {$\cdot$} Ensemble\\
  & Methods {$\cdot$} Hidden Markov Model {$\cdot$} Clustering {$\cdot$} Frequent Pattern Mining\\
  Statistical Modeling & Regression models {$\cdot$} Time series and dynamic
  models {$\cdot$} Ordinary differential equations \\
  & {$\cdot$} Model fitting {$\cdot$} Network simulation\\
\end{tabular}
\end{rSection}

% %----------------------------------------------------------------------------------------
% % Courses
% %----------------------------------------------------------------------------------------

% \begin{rSection}{Courses}

% \begin{tabular}{ @{} >{\bfseries}l @{\hspace{6ex}} l }
  % Graduate Courses & Data Mining {$\cdot$} Statistical Modeling I
  % {$\cdot$} Statistical Models for Big Data {$\cdot$} Time Series \&
  % Dynamic Models {$\cdot$} \\
  % & Regression Analysis\\
  % MOOC & Machine Learning {$\cdot$} Coding the Matrix: Linear Algebra through Computer
  % Science Applications {$\cdot$} Pattern \\
  % & Discovery in Data Mining {$\cdot$} R Programming {$\cdot$}
  % Getting and Cleaning Data {$\cdot$} Exploratory Data Analysis\\
% \end{tabular}
% \end{rSection}

% %----------------------------------------------------------------------------------------
% % Teaching Experience
% %----------------------------------------------------------------------------------------

% \begin{rSection}{Teaching Experience}

% \begin{rSubsection}{Teaching Assistant}{The University of Texas at
  % Austin}{September 2015 - December 2015}
% \begin{rSubsection}{}{}{September 2014 - December 2014}
% \item Mentored two lab sections (48 students) of an undergraduate genetics
  % course and an undergraduate microbiology course;
% \item Got 4.3/5.0 in both course evaluations;
% \item Prepared lab lectures and lab plans;
% \item Graded quizzes, assignments and exams.
% \end{rSubsection}
% \end{rSubsection}
% \end{rSection}

\end{document}
