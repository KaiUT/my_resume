%%%%%%%%%%%%%%%%%%%%%%%%%%%%%%%%%%%%%%%%%
% Resume
% LaTeX Template
% Version 2.0 (8/5/13)
%
% This template has been downloaded from:
% http://www.LaTeXTemplates.com
%
% Original author:
% Trey Hunner (http://www.treyhunner.com/)
%
% Important note:
% This template requires the resume.cls file to be in the same directory as the
% .tex file. The resume.cls file provides the resume style used for structuring the
% document.
%
%%%%%%%%%%%%%%%%%%%%%%%%%%%%%%%%%%%%%%%%%

%----------------------------------------------------------------------------------------
%	PACKAGES AND OTHER DOCUMENT CONFIGURATIONS
%----------------------------------------------------------------------------------------

\documentclass{my_resume} % Use the custom resume.cls style

\usepackage[left=0.7cm,top=0.7cm,right=0.9cm,bottom=0.7cm]{geometry} % Document margins
\newcommand{\tab}[1]{\hspace{.2667\textwidth}\rlap{#1}}
\newcommand{\itab}[1]{\hspace{0em}\rlap{#1}}
\renewcommand{\familydefault}{\sfdefault}
\name{Kai Liu} % Your name
\address{\href{mailto:kai.liu@utexas.edu}{kai.liu@utexas.edu}
  \textbullet{} {512.917.6781}
}

\address{
  GitHub: \href{https://github.com/kaiUT}{github.com/KaiUT} \textbullet{}
  LinkedIn: \href{https://www.linkedin.com/in/kai-liu-utaustin}
  {www.linkedin.com/in/kai-liu-utaustin}
}

\begin{document}

%----------------------------------------------------------------------------------------
%	PROFILE
%----------------------------------------------------------------------------------------

\begin{rSection}{Profile}
  Ph.D. in Computational Biology (expected May 2019) seeking
  \underline{internship opportunities in Data Science/Machine Learning Research
  for}\\ \underline{2018 summer}.
  Solid skills in machine learning, statistical inference.
  Comprehensive computing skills including Python, R, SQL, Hadoop, Spark,
  TensorFlow.
  Excellent problem solving skills in both independent and team environments.
  Skilled presenter of technical materials to both technical and non-technical
  audiences.
  Quick, thorough and effective learner.
\end{rSection}

%----------------------------------------------------------------------------------------
%	EDUCATION SECTION
%----------------------------------------------------------------------------------------

\begin{rSection}{Education}

{\bf Ph.D. in Computational Biology} {$\diamond$} {\em The University of
Texas at Austin, Austin, TX} \hfill {Expected May 2019} \\
{\bf M.S. in Microbiology} {$\diamond$} {\em Huazhong Agricultural University,
Wuhan, China} \hfill {Grad. June 2013} \\
{\bf B.S. in Biotechnology} {$\diamond$} {\em Huazhong Agricultural University,
Wuhan, China} \hfill {Grad. June 2011}
\end{rSection}

%----------------------------------------------------------------------------------------
%	WORK EXPERIENCE SECTION
%----------------------------------------------------------------------------------------

\begin{rSection}{Work Experience}

  \begin{rSubsection}{Machine Learning Intern}{QuintilesIMS, Plymouth Meeting,
    PA}{June 2017 - August 2017}

  \item[] \textbf{Predicted Quality of Investigators in Future Clinical Trials
    $|$ Python, Spark}
  \item Predicted outliers of investigators per Key Risk Indicator using
    \underline{distribution based approach};
  \item Built multiple machine learning models (\underline{Lasso Regression, Neural
    Network, Random Forests}) to predict the quality of investigators in a
    future study, which is one of the core projects in the investigator
    recommender system.
  \end{rSubsection}

  \begin{rSubsection}{Graduate Research Assistant}{The University of Texas at
    Austin, Austin, TX}{December 2014 - Present}

  \item[] \textbf{Developed Infectious Diseases Surveillance App $|$ Python}
  \item Developed a \underline{regression model} and a \underline{Multivariate
      Exponentially Weighted Moving Average (MEWMA)} model to detect emerging
      outbreaks with an accuracy of 0.9;
  \item Combined above models with \underline{stepwise variable selection
    algorithms} to select best data sources for infectious diseases
    surveillance (more than 400 data sources in total);
  \item Built up data pipeline to automate the process of retrieving and
    cleaning data from CDC, RSS feed, Google Trends, Wikipedia, Twitter etc;
    integrated the App into Cloud Ecosystem.

  % \item[] \textbf{Assessed Real-time Zika Risk in the State of Texas $|$ R}
  % \item Collaborated with other researchers in developing a \underline{branching
      % process model framework} that captures variation and uncertainty in Zika
      % case reporting, importations, and transmission;
  % \item Applied the framework to assess county-level epidemic risk throughout
    % Texas.

  % \underline{Developed Mathematical Models for Infectious Diseases}
  % \item Developed Ordinary Differential Equations framework for infectious diseases
    % eliminating the assumption that all individuals in a population have the
    % same number of contacts, based on models published previously.
  % \item Simulated infectious diseases spreading on contact networks using
    % multiple algorithms.
% \item Publication in preparation.
  \end{rSubsection}
\end{rSection}

%----------------------------------------------------------------------------------------
% Publications
%----------------------------------------------------------------------------------------

% \begin{rSection}{Publications}
  % \begin{itemize}
    % \setlength\itemsep{-1.6em}
    % \setlength\leftskip{-1.2em}
    % \item[$\cdot$] {\bf Liu K}, Miller JC, Meyers LA. Effects of Directed and Clustered Contact
    % Patterns on Infectious Disease Dynamics. {\em In preparation}.\\
    % \item[$\cdot$] {\bf Liu K}, Srinivasan R, Ertem Z, Meyers LA. Optimizing Early Detection
    % of Emerging Outbreaks. {\em Submitted}.\\
    % \item[$\cdot$] Castro LC*, Fox SJ*, Chen X, {\bf Liu K}, Bellan SE, Dimitrov NB, Galvani AP,
    % Meyers LA. Assessing Real-time Zika Risk in the United States. {\em BMC
    % Infectious Diseases}. DOI: 10.1186/s12879-017-2394-9.
  % \end{itemize}
% \end{rSection}

%----------------------------------------------------------------------------------------
% Projects
%----------------------------------------------------------------------------------------

\begin{rSection}{Personal Projects}
  \begin{rSubsection}{Being Involved in Building an Open Source Software to
      Detect Lung Cancer $|$ Python, TensorFlow}{}{August 2017 - Present}
    \item Contributing to improve the \underline{3D Convolutional Neural Network} that identifies
    locations of nodules in scans;
  \item Contributing to improve the \underline{3D Convolutional Neural Network} to find the
    boundaries of nodules in scans.
  \end{rSubsection}

  \begin{rSubsection}{Developing a R Package for Big Data Analysis $|$ R \&
    Rcpp}{}{December 2016 - Present}
  \item Implementing following algorithms in the package: Stochastic gradient
    descent using line search and quasi-Newton methods to determine step size
    {$\cdot$} The lasso {$\cdot$} The proximal gradient method {$\cdot$}
    Laplacian smoothing solved by sparse Cholesky/LU, the Gauss-Seidel method,
    the Jacobi iterative method, and conjugate gradient method {$\cdot$} Graph
    fused lasso solved by Alternating Direction Method of Multipliers (ADMM)
    {$\cdot$} Sparse matrix factorization.
  \end{rSubsection}

  \begin{rSubsection}{Predicted the Direction of Exchange-Traded Fund
    (ETF) movement $|$ Python}{}{April 2017 - May 2017}
  \item Retrieved nine histrorical ETF sector datasets from Yahoo Finance;
  \item Implemented \underline{Logistic regression}, \underline{Ridge \& Lasso
    regression}, and \underline{Artificial Neural Network} to predict the
    direction of ETF movement;
  \item Achieved an \underline{accuracy of 55\% $\sim$ 60\%} for predicting
    nine ETF sectors movement; and the trading strategy based on my prediction
    \underline{outperforms} baseline strategies.
  \end{rSubsection}

  % \begin{rSubsection}{Denoised GPS Data by Applying Kalman Filter $|$ R}{}{}
  % \item Implemented \underline{Kalman filter}, and smoothed GPS data collected from
    % a vehicle cruising around campus (814458 samples).
  % \end{rSubsection}

  \begin{rSubsection}{Predicted Yelp Rating Based on User Review Enhanced
    Collaborative Filtering $|$ R}{}{September 2015 - December 2015}
  \item Extracted user opinions from restaurants dataset from Yelp ($\sim$10GB)
    using \underline{Stanford coreNLP tool};
  \item Developed a \underline{new Collaborative Filtering-based method} to
    improve the accuracy of user's rating prediction and solve the sparseness
    of dataset by combining item's features and user opinions from all reviews;
  \item \underline{Improved the prediction accuracy by 4.23\%} compared to the
    traditional KNN method, and the \underline{coverage is 100\%}.
  \end{rSubsection}

% \begin{rSubsection}{Forecasted Tourism Earnings of United Kingdom}{}{October
  % 2015 - December 2015}
  % \item Predicted Tourism Earnings of UK using a dynamic linear
    % regression model and Forward Filtering and Backward Sampling algorithm.
% \end{rSubsection}

% \begin{rSubsection}{Statistical Modeling}{}{November 2014 - December 2014}
  % \item Applied a regression model to a dataset to determine 1) factors related
  % to 12 month weight loss, and 2) whether an intervention was effective in
  % increasing weight loss by using both frequentist and Bayesian inference
  % methods.
% \end{rSubsection}
\end{rSection}

%----------------------------------------------------------------------------------------
% Skills
%----------------------------------------------------------------------------------------

\begin{rSection}{Skills}

\begin{tabular}{ @{} >{\bfseries}l @{\hspace{6ex}} l }
  Programming & Fluency in Python(NumPy, SciPy, pandas, scikit-learn),
  R, Git {$\cdot$} Familiar with MATLAB, Linux, LaTex {$\cdot$}\\
  & Experience in SQL, Hadoop, Spark, TensorFlow, C++ \\
  Machine Learning & Deep Neural Network {$\cdot$} Regression with regularization {$\cdot$}
  Support Vector Machine {$\cdot$} Random Forests {$\cdot$} Hidden\\
  & Markov Model {$\cdot$} Clustering {$\cdot$} Time series and dynamic
  models {$\cdot$} Frequent Pattern Mining {$\cdot$} Natural\\
  & Language Processing {$\cdot$} Image Processing \\
\end{tabular}
\end{rSection}

% %----------------------------------------------------------------------------------------
% % Courses
% %----------------------------------------------------------------------------------------

% \begin{rSection}{Courses}

% \begin{tabular}{ @{} >{\bfseries}l @{\hspace{6ex}} l }
  % Graduate Courses & Data Mining {$\cdot$} Statistical Modeling I
  % {$\cdot$} Statistical Models for Big Data {$\cdot$} Time Series \&
  % Dynamic Models {$\cdot$} \\
  % & Regression Analysis\\
  % MOOC & Machine Learning {$\cdot$} Coding the Matrix: Linear Algebra through Computer
  % Science Applications {$\cdot$} Pattern \\
  % & Discovery in Data Mining {$\cdot$} R Programming {$\cdot$}
  % Getting and Cleaning Data {$\cdot$} Exploratory Data Analysis\\
% \end{tabular}
% \end{rSection}

% %----------------------------------------------------------------------------------------
% % Teaching Experience
% %----------------------------------------------------------------------------------------

% \begin{rSection}{Teaching Experience}

% \begin{rSubsection}{Teaching Assistant}{The University of Texas at
  % Austin}{September 2015 - December 2015}
% \begin{rSubsection}{}{}{September 2014 - December 2014}
% \item Mentored two lab sections (48 students) of an undergraduate genetics
  % course and an undergraduate microbiology course;
% \item Got 4.3/5.0 in both course evaluations;
% \item Prepared lab lectures and lab plans;
% \item Graded quizzes, assignments and exams.
% \end{rSubsection}
% \end{rSubsection}
% \end{rSection}

\end{document}
\end{document}
