%%%%%%%%%%%%%%%%%%%%%%%%%%%%%%%%%%%%%%%%%
% Resume
% LaTeX Template
% Version 2.0 (8/5/13)
%
% This template has been downloaded from:
% http://www.LaTeXTemplates.com
%
% Original author:
% Trey Hunner (http://www.treyhunner.com/)
%
% Important note:
% This template requires the resume.cls file to be in the same directory as the
% .tex file. The resume.cls file provides the resume style used for structuring the
% document.
%
%%%%%%%%%%%%%%%%%%%%%%%%%%%%%%%%%%%%%%%%%

%----------------------------------------------------------------------------------------
%	PACKAGES AND OTHER DOCUMENT CONFIGURATIONS
%----------------------------------------------------------------------------------------

\documentclass{my_resume} % Use the custom resume.cls style

\usepackage[left=0.7cm,top=0.7cm,right=0.9cm,bottom=0.7cm]{geometry} % Document margins
\newcommand{\tab}[1]{\hspace{.2667\textwidth}\rlap{#1}}
\newcommand{\itab}[1]{\hspace{0em}\rlap{#1}}
\renewcommand{\familydefault}{\sfdefault}
\name{Kai Liu} % Your name
\address{\href{mailto:kai.liu@utexas.edu}{kai.liu@utexas.edu}
  \textbullet{} {512.917.6781} \textbullet{}
  GitHub://\href{https://github.com/KaiUT}{KaiUT} \textbullet{}
  LinkedIn://\href{https://www.linkedin.com/in/kai-liu-27200774}{Kai Liu}
}

\begin{document}

%----------------------------------------------------------------------------------------
%	PROFILE
%----------------------------------------------------------------------------------------

\begin{rSection}{Profile}
% \begin{rSubsection}{ }{}{ }
\renewcommand\labelitemi{$\cdot$}
\begin{itemize}
  \item Ph.D in Computational Biology (expected August 2018);
  \vspace{-1.7em}\\
  \item Ample Skills in data science:
  \vspace{-1.7em}\\
    \begin{itemize}
      \item Experience in data mining, machine learning, statistical inference,
        and big data analysis;
      \vspace{-1.7em}\\
      \item Comprehensive technical/computing skills includes Python, R, Git,
        C++, SQL, Spark;
      \vspace{-1.7em}\\
    \end{itemize}
  \item Excellent problem solving skills in both independent and team
    environments;
  \vspace{-1.7em}\\
  \item Skilled presenter of technical material to both technical and
    non-technical audiences;
  \vspace{-1.7em}\\
  \item Quick, thorough and effective learner.
\end{itemize}
% \end{rSubsection}
\end{rSection}

%----------------------------------------------------------------------------------------
%	EDUCATION SECTION
%----------------------------------------------------------------------------------------

\begin{rSection}{Education}

{\bf Ph.D in Computational Biology} {$\diamond$} {\em The University of
Texas at Austin, Austin, TX} \hfill {Expected August 2018} \\
% \vspace{0.3em}\\
{\bf M.S in Microbiology} {$\diamond$} {\em Huazhong Agricultural University,
Wuhan, China} \hfill {Grad. June 2013} \\
% \vspace{0.3em}\\
{\bf B.S in Biotechnology} {$\diamond$} {\em Huazhong Agricultural University,
Wuhan, China} \hfill {Grad. June 2011}
\end{rSection}

%----------------------------------------------------------------------------------------
%	RESEARCH EXPERIENCE SECTION
%----------------------------------------------------------------------------------------

\begin{rSection}{Research Experience}

\begin{rSubsection}{Graduate Research Assistant}{The
    University of Texas at Austin}{December 2014 - Present}
  \underline{Developing Infectious Diseases Surveillance App in Python}
  \item Retrieved and cleaned infectious diseases related data from Google
    Trends, Wikipedia, WordPress etc;
  \item Developed a regression model and a Multivariate Exponentially Weighted
    Moving Average(MEWMA) model to detect infectious disease outbreaks using
    multiple data sources, in collaboration with a mathematician;
  \item Assessing the algorithms and optimizing data sources on infectious
    diseases in different regions (using 552 time series data sources);
  \item Improving performance and speed of the algorithms;
  \item Connecting algorithms with the App back-end and front-end, and
    integrating the App into Cloud Ecosystem, in collaboration with a front-end
    engineer.

  \underline{Assessed Real-time Zika Risk in the State of Texas}
  \item Collaborated with other researchers in developing a branching process
    model framework that captures variation and uncertainty in Zika case
    reporting, importations, and transmission;
  \item Applied the framework to assess county-level epidemic risk throughout
    Texas.

  \underline{Developed Mathematical Models for Infectious Diseases}
  \item Developed Ordinary Differential Equations framework for infectious diseases
    eliminating the assumption that all individuals in a population have the
    same number of contacts, based on models published previously.
  \item Simulated infectious diseases spreading on contact networks using
    multiple algorithms.
% \item Publication in preparation.
\end{rSubsection}
\end{rSection}

%----------------------------------------------------------------------------------------
% Course Projects
%----------------------------------------------------------------------------------------

\begin{rSection}{Course Projects}

\begin{rSubsection}{Denoised GPS Data by Applying Kalman Filter}{}{October
    2016 - December 2016}
  \item Implemented Kalman filter in R, and smoothed GPS data collected from a
  vehicle cruising around campus (814458 samples).
\end{rSubsection}

\begin{rSubsection}{Predicted Yelp Rating Based on User Review Enhanced
  Collaborative Filtering}{}{September 2015 - December 2015}
  \item Developed a new Collaborative Filtering-based method to improve the
    accuracy of user's rating prediction and solve the sparseness of dataset by
    combining item's features and user opinions from all reviews;
  \item Applied the new method to predict user ratings using restaurants dataset
    from Yelp ($\sim$10GB). Its performance is 4.23\% better than that of
    traditional KNN method, and its coverage is 100\%.
\end{rSubsection}

\begin{rSubsection}{Forecasted Tourism Earnings of United Kingdom}{}{October
  2015 - December 2015}
  \item Predicted Tourism Earnings of UK using a dynamic linear
    regression model and Forward Filtering and Backward Sampling algorithm.
\end{rSubsection}

\begin{rSubsection}{Statistical Modeling}{}{November 2014 - December 2014}
  \item Applied a regression model to a dataset to determine 1) factors related
  to 12 month weight loss, and 2) whether an intervention was effective in
  increasing weight loss by using both frequentist and Bayesian inference
  methods.
\end{rSubsection}
\end{rSection}

%----------------------------------------------------------------------------------------
% Skills
%----------------------------------------------------------------------------------------

\begin{rSection}{Skills}

\begin{tabular}{ @{} >{\bfseries}l @{\hspace{6ex}} l }
  Statistical Modeling & Regression models {$\cdot$} Time series and
  dynamic models {$\cdot$} Hypothesis testing and confidence \\
  & interval {$\cdot$} Data fitting {$\cdot$} Ordinary differential
  equations {$\cdot$} Network simulation\\
  Big Data Analysis & Online learning {$\cdot$} Regularization and sparsity in
  statistical models {$\cdot$} Matrix factorization {$\cdot$}\\
  & Spatial smoothing {$\cdot$} Principal component analysis and
  dimensionality reduction\\
  Data Mining \& Machine Learning & Regression {$\cdot$} Classification
  {$\cdot$} Clustering {$\cdot$} Frequent Pattern Mining\\
  Programming & Fluency in Python(NumPy, SciPy, Matplotlib, pandas,
  scikit-learn), R, Git {$\cdot$} Familiar with \\
  & MATLAB, Linux, LaTex {$\cdot$} Experience in C++, SQL, Spark\\
\end{tabular}
\end{rSection}

%----------------------------------------------------------------------------------------
% Courses
%----------------------------------------------------------------------------------------

\begin{rSection}{Courses}

\begin{tabular}{ @{} >{\bfseries}l @{\hspace{6ex}} l }
  Graduate Courses & Data Mining {$\cdot$} Statistical Modeling I
  {$\cdot$} Statistical Models for Big Data {$\cdot$} Time Series \&
  Dynamic Models {$\cdot$} \\
  & Regression Analysis\\
  MOOC & Machine Learning {$\cdot$} Coding the Matrix: Linear Algebra through Computer
  Science Applications {$\cdot$} Pattern \\
  & Discovery in Data Mining {$\cdot$} R Programming {$\cdot$}
  Getting and Cleaning Data {$\cdot$} Exploratory Data Analysis\\
\end{tabular}
\end{rSection}

%----------------------------------------------------------------------------------------
% Teaching Experience
%----------------------------------------------------------------------------------------

\begin{rSection}{Teaching Experience}

\begin{rSubsection}{Teaching Assistant}{The University of Texas at
  Austin}{September 2015 - December 2015}
\begin{rSubsection}{}{}{September 2014 - December 2014}
\item Mentored two lab sections (48 students) of an undergraduate genetics
  course and an undergraduate microbiology course;
\item Got 4.3/5.0 in both course evaluations;
\item Prepared lab lectures and lab plans;
\item Graded quizzes, assignments and exams.
\end{rSubsection}
\end{rSubsection}
\end{rSection}

\end{document}
