%%%%%%%%%%%%%%%%%%%%%%%%%%%%%%%%%%%%%%%
% Deedy - One Page Two Column Resume
% LaTeX Template
% Version 1.1 (30/4/2014)
%
% Original author:
% Debarghya Das (http://debarghyadas.com)
%
% Original repository:
% https://github.com/deedydas/Deedy-Resume
%
% IMPORTANT: THIS TEMPLATE NEEDS TO BE COMPILED WITH XeLaTeX
%
% This template uses several fonts not included with Windows/Linux by
% default. If you get compilation errors saying a font is missing, find the line
% on which the font is used and either change it to a font included with your
% operating system or comment the line out to use the default font.

%%%%%%%%%%%%%%%%%%%%%%%%%%%%%%%%%%%%%%

% TODO:
% 1. Integrate biber/bibtex for article citation under publications.
% 2. Figure out a smoother way for the document to flow onto the next page.
% 3. Add styling information for a "Projects/Hacks" section.
% 4. Add location/address information
% 5. Merge OpenFont and MacFonts as a single sty with options.

%%%%%%%%%%%%%%%%%%%%%%%%%%%%%%%%%%%%%%
%
% CHANGELOG:
% v1.1:
% 1. Fixed several compilation bugs with \renewcommand
% 2. Got Open-source fonts (Windows/Linux support)
% 3. Added Last Updated
% 4. Move Title styling into .sty
% 5. Commented .sty file.
%
%%%%%%%%%%%%%%%%%%%%%%%%%%%%%%%%%%%%%%%
%
% Known Issues:
% 1. Overflows onto second page if any column's contents are more than the
% vertical limit
% 2. Hacky space on the first bullet point on the second column.
%
%%%%%%%%%%%%%%%%%%%%%%%%%%%%%%%%%%%%%%

\documentclass[]{deedy-resume-openfont}


\begin{document}

%%%%%%%%%%%%%%%%%%%%%%%%%%%%%%%%%%%%%%
%
%     Resume for which company
%
%%%%%%%%%%%%%%%%%%%%%%%%%%%%%%%%%%%%%%
% NOTE: need to change for each target company
\forcompany{XXXXXXXXXX}

%%%%%%%%%%%%%%%%%%%%%%%%%%%%%%%%%%%%%%
%
%     LAST UPDATED DATE
%
%%%%%%%%%%%%%%%%%%%%%%%%%%%%%%%%%%%%%%
\lastupdated

%%%%%%%%%%%%%%%%%%%%%%%%%%%%%%%%%%%%%%
%
%     TITLE NAME
%
%%%%%%%%%%%%%%%%%%%%%%%%%%%%%%%%%%%%%%


\namesection{Kai}{Liu}
{
  \href{mailto:kai.liu@utexas.edu}{kai.liu@utexas.edu} \textbullet{}
  512.917.6781 \textbullet{} GitHub://
  \href{https://github.com/KaiUT}{\custombold{KaiUT}} \textbullet{} LinkedIn://
  \href {https://www.linkedin.com/in/kai-liu-27200774}{\custombold{Kai
  Liu}}
}

%%%%%%%%%%%%%%%%%%%%%%%%%%%%%%%%%%%%%%
%
%     COLUMN ONE
%
%%%%%%%%%%%%%%%%%%%%%%%%%%%%%%%%%%%%%%

\begin{minipage}[t]{0.3\textwidth}

%%%%%%%%%%%%%%%%%%%%%%%%%%%%%%%%%%%%%%
%     EDUCATION
%%%%%%%%%%%%%%%%%%%%%%%%%%%%%%%%%%%%%%

\section{Education}

\subsection{The University of Texas at Austin}
\descript{Ph.D in Computational Biology}
\location{Expected August 2018 | Austin, TX}
\sectionsep

\subsection{Huazhong Agricultural University}
\descript{M.S in Microbiology}
\location{Grad. June 2013 | Wuhan, China}
\sectionsep

\descript{B.S in Biotechnology}
\location{Grad. June 2011 | Wuhan, China}
\sectionsep

%%%%%%%%%%%%%%%%%%%%%%%%%%%%%%%%%%%%%%
%     SKILLS
%%%%%%%%%%%%%%%%%%%%%%%%%%%%%%%%%%%%%%

\section{Skills}
\subsection{Statistical Modeling}
Regression models \textbullet{}
Time series and dynamic models \textbullet{}
Hypothesis testing and confidence interval \textbullet{}
Data fitting (Least square, maximum likelihood, Bayesian methods) \textbullet{}
Ordinary differential equations (ODEs) \textbullet{}
Network simulation
\sectionsep

\subsection{Big Data Analysis}
Online learning \textbullet{}
Regularization and sparsity in statistical models \textbullet{}
Matrix factorization \textbullet{}
Spatial smoothing \textbullet{}
Principal component analysis and dimensionality reduction
\sectionsep

\subsection{Data Mining and Machine Learning}
Regression \textbullet{}
Classification \textbullet{}
Clustering \textbullet{}
Frequent Pattern Mining
\sectionsep

\subsection{Programming}
Fluency in Python, R \textbullet{}
Fluency in Git, Vim, Linux \textbullet{}
Familiar with C++, MATLAB, LaTex
\sectionsep

%%%%%%%%%%%%%%%%%%%%%%%%%%%%%%%%%%%%%%
%     COURSEWORK
%%%%%%%%%%%%%%%%%%%%%%%%%%%%%%%%%%%%%%

\section{Coursework}
\subsection{Graduate}
% Data Mining; \\
% Statistical Modeling I; \\
% Statistical Models for Big Data; \\
% Time Series \& Dynamic Models; \\
% Regression Analysis. \\
Data Mining \textbullet{}
Statistical Modeling I \textbullet{}
Statistical Models for Big Data \textbullet{}
Time Series \& Dynamic Models \textbullet{}
Regression Analysis
\sectionsep

\subsection{MOOC}
% Machine Learning; \\
% Coding the Matrix: Linear Algebra through Computer Science Applications; \\
% Pattern Discovery in Data Mining; \\
% R Programming; \\
% Getting and Cleaning Data; \\
% Exploratory Data Analysis. \\
Machine Learning \textbullet{}
Coding the Matrix: Linear Algebra through Computer Science Applications
\textbullet{}
Pattern Discovery in Data Mining; \textbullet{}
R Programming \textbullet{}
Getting and Cleaning Data \textbullet{}
Exploratory Data Analysis
\sectionsep


%%%%%%%%%%%%%%%%%%%%%%%%%%%%%%%%%%%%%%
%
%     COLUMN TWO
%
%%%%%%%%%%%%%%%%%%%%%%%%%%%%%%%%%%%%%%

\end{minipage}
\hfill
\begin{minipage}[t]{0.66\textwidth}

%%%%%%%%%%%%%%%%%%%%%%%%%%%%%%%%%%%%%%
%     Research Projects
%%%%%%%%%%%%%%%%%%%%%%%%%%%%%%%%%%%%%%

\section{Research Projects}

\runsubsection{Developing Surety BioEvent App} \\
% \descript{| DTRA + UT Austin }
\location{December 2015 – Present}
% Developing an Python application for the evaluation and optimal inclusion of multiple
% data sources into bioevent surveillance for Situational Awareness (SA), Early
% Detection (ED) and Forecasting, with a team of mathematicians, statisticians
% and computer engineers.\\
\vspace{\topsep} % Hacky fix for awkward extra vertical space
\begin{tightemize}
\item Retrieved and cleaned Influenza- and Dengue-related data from Athena
  Health, Google Trends, Wikipedia etc;
\item Collaborated with a mathematician in developing a new Early Event
  Detection (ED) algorithm to detect infectious disease outbreaks using
  multiple data sources;
\item Assessing the Situational Awareness (SA) and ED algorithms on Influenza
  in US (using 452 time series data sources) and Dengue in Puerto Rico (using
    100 time series data sources);
\item Improving performance and speed of the ED algorithm;
\item Collaborating with a front-end engineer to connect algorithms with
  the App back-end and front-end, and integrate the App into Biosurveillance
  Ecosystem.
% \item Publication in preparation.
\end{tightemize}
\sectionsep

\runsubsection{Assessed Real-time Zika Risk in the State of Texas} \\
% \descript{| UT Austin}
\location{March 2016 – May 2016}
\begin{tightemize}
\item Collaborated with other researchers in developing a branching process
  model framework that captures variation and uncertainty in Zika case
  reporting, importations, and transmission;
\item Applied the framework to assess county-level epidemic risk throughout
  Texas.
\end{tightemize}
\sectionsep

\runsubsection{Developed Edge-based Mathematical Models for Infectious Diseases} \\
% \descript{| Open Source Contributor \& Team Leader}
\location{December 2014 – November 2015}
\begin{tightemize}
\item Developed susceptible-exposed-infectious-recovered mathematical models for
infectious diseases eliminating the assumption that all individuals in a
population have the same number of contacts, based on models published
previously;
\item Compared results from these models with that from infectious diseases spreading
  simulations on contact networks.
% \item Publication in preparation.
\end{tightemize}
\sectionsep

%%%%%%%%%%%%%%%%%%%%%%%%%%%%%%%%%%%%%%
%     Course Projects
%%%%%%%%%%%%%%%%%%%%%%%%%%%%%%%%%%%%%%

\section{Course Projects}
\runsubsection{Denoised GPS Data by Applying Kalman Filter} \\
% \descript{| head undergrad research}
\location{October 2016 – December 2016}
\begin{tightemize}
\item Implemented Kalman filter in R; and smoothed GPS data collected from a
  police vehicle cruising around campus (814458 samples).
\end{tightemize}
\sectionsep

\runsubsection{Predicted Yelp Rating Based on User Review Enhanced
Collaborative Filtering} \\
% \descript{| head undergraduate researcher}
\location{September 2015 – December 2015}
\begin{tightemize}
\item Developed a new Collaborative Filtering-based method to improve the
  accuracy of user's rating prediction and solve the sparseness of dataset by
  combining item's features and user opinions from all reviews;
\item Applied the new method to predict user ratings using restaurants dataset
  from Yelp (5GB). Its performance is 4.23\% better than that of traditional KNN
  method, and its coverage is 100\%.
\end{tightemize}
\sectionsep

\runsubsection{Forecasted Tourism Earnings of United Kingdom} \\
% \descript{| head undergraduate researcher}
\location{October 2015 – December 2015}
\begin{tightemize}
\item Predicted Tourism Earnings of UK using a dynamic linear
  regression model and Forward Filtering and Backward Sampling algorithm in
  R.
\end{tightemize}
\sectionsep

\runsubsection{Statistical Modeling} \\
% \descript{| Head Undergraduate Researcher}
\location{November 2014 – December 2014}
\begin{tightemize}
\item Analyzed a dataset to determine 1) factors related to 12 month weight
loss, and 2) whether an intervention was effective in increasing weight loss by
applying both frequentist and Bayesian inference methods.
\end{tightemize}
\sectionsep

\end{minipage}
\end{document}  \documentclass[]{article}
